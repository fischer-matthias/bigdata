% !TEX root = ../document.tex
\section{Einleitung}
\label{sec:Einleitung}
Image Recognition an sich ist zwar keine Neuheit und besteht schon seit dem Ende des 20. Jahrhunderts, hat allerdings vor allem in den letzten Jahren im Kontext von Begriffen wie bspw. Big Data, Machine Learning oder Deep Learning zunehmend an Bedeutung gewonnen. Dies lässt sich auch daran erkennen, dass viele Unternehmen, darunter auch bedeutende Unternehmen wie z. B. Google, Facebook, Microsoft, Apple oder Pinterest, beträchtliche Ressourcen in die Erforschung und Entwicklung von Software zur Bilderkennung investieren. Dies ist nicht zuletzt auch darauf zurückzuführen, dass sich mit Hilfe der Bilderkennung Anwendungen in verschiedenen Bereichen wie gezielter Werbung, angepassten Suchen oder „smarten“ Fotobibliotheken realisieren lassen.\footnote{\cite{ImageRecognition}}

Da die Funktionalität auf der untersten Ebene solcher Bilderkennungssoftware auf sehr komplexen informationstechnischen und mathematischen Algorithmen beruht, nehmen viele Menschen diese als gegeben hin. Ziel dieser Arbeit ist es somit, anhand von sog. Convolutional Neural Networks einen Überblick über die Funktionsweise der Bilderkennung (Image Recognition) und Bildklassifizierung (Image Classification) zu geben und anhand eines praktischen Beispiels in die Software TensorFlow zum Trainieren und Erkennen von Bildern einzuführen.

Dazu wird zunächst darauf eingegangen was genau Image Recognition ist und wie dies in den übergeordneten Kontext von Pattern Recognition, Machine Learning und Big Data einzuordnen ist. Danach wird die Software TensorFlow vorgestellt und darauf eingegangen, wie das Prinzip des Image Retrainings und Image Recognitionings innerhalb dieser funktioniert. Im Anschluss daran wird auf das Convolutional Neural Network als Möglichkeit zur Bildanalyse eingegangen und erläutert, wie dieses aufgebaut ist und funktioniert und wie Image Retraining und Image Recognition mit Hilfe dessen umgesetzt werden können. Darauffolgend beginnt der Praxisteil, in dem zunächst darauf eingegangen wird, wie TensorFlow installiert und eingerichtet wird. Dann wir dargestellt, wie TensorFlow mit Hilfe seiner Python-API verwendet werden kann, um Image Retraining und Image Recognition durchzuführen.

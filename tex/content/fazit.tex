% !TEX root = ../document.tex
\section{Fazit}
\label{sec:Fazit}
Letztlich lässt sich sagen, dass TensorFlow eine geeignete Möglichkeit für das Image Recognitioning darstellt. Die Verwendung fertiger Skripte macht eine schnelle Anwendung möglich und ist aufgrund der einfachen Verwendung auch schnell erlernbar. Durch die mitgelieferten Funktionen ist auch die eigene Entwicklung eines Convolutional Neural Networks problemlos möglich und dadurch, dass TensorFlow Open-Source ist, können auch bestehende Funktionen nach Belieben angepasst werden. TensorFlow macht es so möglich Machine Learning vor allem mit Hinblick auf Bildverarbeitung trotz komplexer mathematischer Theorie einfach verwenden zu können. Auch wenn das Convolutional Neural Network bereits eine sehr ausgereifte Art der Bildverarbeitung darstellt, wird dennoch weiterhin an neuen und effizienteren Methoden geforscht und auch versucht bestehende Möglichkeiten wie das Convolutional Neural Network stets weiterzuentwickeln, weshalb Interessierte sich in diesem sehr aktuellen Bereich des Image Recognitionings stets weiter informieren sollten.